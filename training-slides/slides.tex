\documentclass[aspectratio=169]{beamer}

\usetheme{permedcoe}

\usepackage{qrcode}
\usepackage{amsmath}

\usetikzlibrary{positioning, fit, decorations.pathmorphing, decorations.pathreplacing, arrows.meta, calc}
\tikzstyle{lwd}=[line width=.2ex]
\tikzstyle{blk}=[rectangle,font=\condensed,text width=5em, align=center, inner sep=1ex]
\tikzstyle{sblk}=[rectangle,font=\condensed\scriptsize,text width=4em, align=center, inner sep=1ex]
\tikzstyle{arr}=[-{>[length=1ex]}]
\def\mkMdl#1{%
  \draw[decorate, decoration=zigzag] (#1.north west) to (#1.north east);
  \draw[rounded corners=.5ex] (#1.north east) to (#1.south east) to (#1.south west) to (#1.north west);
}

\def\mkWrp#1{%
  \draw[decorate, decoration=zigzag] (#1.north west) to (#1.north east);
  \draw[decorate, decoration=zigzag] (#1.south west) to (#1.south east);
  \draw (#1.north east) to (#1.south east);
  \draw (#1.north west) to (#1.south west);
}

\def\mkAna#1{%
  \draw[decorate, decoration=zigzag] (#1.south west) to (#1.south east);
  \draw[rounded corners=.5ex] (#1.south east) to (#1.north east) to (#1.north west) to (#1.south west);
}

\title{Systematising complex and combined metabolic analyses with COBREXA.jl}
\author{Mirek Kratochvíl (Uni.Lu), Daniel Thomas Lopez (EMBL EBI)}
\date{Nov 15\textsuperscript{th} 2022}

%\setbeamertemplate{frame footer}[authordate]
\setbeamertemplate{frame footer}[custom]{Systematising analyses with COBREXA.jl\quad\insertdate}
\setbeamertemplate{standout background}[permedcoe]

\begin{document}
\maketitle

\setcounter{section}{-1}
\section{Preparatory session}

\begin{frame}{Check-list}
\begin{itemize}
\item Julia is installed
\item COBREXA is installed
\begin{enumerate}
\item start Julia
\item press \textbf{\Large\ttfamily ]} to activate packaging mode
\item type \texttt{add COBREXA}
\item return to normal mode using backspace
\end{enumerate}
\item at least one solver is installed (preferably GLPK)
\item packages compile all right
\end{itemize}
\end{frame}

\begin{frame}[fragile]{Test data}
\begin{center}
\verb|http://bigg.ucsd.edu/static/models/e_coli_core.json|

(search engines: `BiGG E coli Core JSON')
\end{center}
\end{frame}

\begin{frame}[fragile]{Test}
\begin{verbatim}
using COBREXA, GLPK

download(
  "http://bigg.ucsd.edu/static/models/e_coli_core.json",
  "e_coli_core.json")

model = load_model("e_coli_core.json")

fluxes = flux_balance_analysis_dict(model, GLPK.Optimizer)
\end{verbatim}
\end{frame}

{
\renewcommand{\lastslidethanks}{See you on Nov 15\textsuperscript{th}!}
\makelastslide{}
}

\section{What's and Why's}

\begin{frame}{Constraint-based modeling (recap)}
\begin{columns}
\begin{column}{.7\linewidth}
\begin{enumerate}
\item Have some biochemical knowledge about processes in organisms
\item Summarize the knowledge as a linear constrained model
  \begin{itemize}
    \item reactions convert metabolites to other metabolites
    \item atoms are preserved
  \end{itemize}
\item Gather observations about your model using linear optimization machinery
\end{enumerate}

Main systematic problems:
\begin{itemize}
\item What is the best knowledge representation?
\item How precise is the knowledge?
\end{itemize}
\end{column}
\begin{column}{.3\linewidth}
\begin{align*}
\text{find}\quad\arg\max_x &\phantom{=} c^T x \\
\text{such that}\quad S x &= 0 \\
x &\geq u \\
x &\leq l \\
\end{align*}
\end{column}
\end{columns}
\end{frame}

\begin{frame}{COBRA tool development}
\begin{description}
\item[Old times] Build the model from what you remember from school\&lab; directly use the LP solver.
\item[Boxed analyses] Use precise model formats and several streamlined reconstruction procedures, enjoy some analyses implemented in COBRA toolboxes (COBRApy, matlab, \dots).
\item[Extensibility] Use arbitrary model formats, construct any procedures and analyses from sensible reusable building blocks.
\end{description}
\end{frame}

\begin{frame}{Main topics for today}
\begin{itemize}
\item Avoid long fragile analysis scripts
\item Learn to `package' and `compose' things that happen in metabolic modeling
\item Gain some side software-engineering benefits (efficiency, parallelization\&reentrancy, reproducibility, repurposability, transparency).
\end{itemize}
\end{frame}

\section{COBREXA.jl Design Primer}

\begin{frame}{What is a model?}
\textbf{Traditional answer:} ``Representation of an organism that may or may not have genes, reactions, metabolites, biopolymers, annotations, gene-reaction associations, enzyme amounts, cell signaling state, \dots, preferably reflecting some reality and solving just right''

\bigskip
\textbf{COBREXA.jl view:} ``Anything we can convert to an annotated linear model''
\end{frame}

\begin{frame}[fragile]{Model types}
\framesubtitle{Idea: Let's not care about the actual model type}
\begin{columns}
\begin{column}{.5\linewidth}
Specialized model types:

\begin{itemize}\tiny
\item \verb|SBMLModel|
\item \verb|MATModel|
\item \verb|JSONModel|
\item \verb|HDF5Model|
\item \verb|StandardModel| (COBRApy-like)
\item \verb|CoreModel| (core linear-algebraic model)
\item \verb|GeckoModel|
\item \verb|SMomentModel|
\item \dots
\end{itemize}

Interfaces (abstract):
\begin{itemize}\footnotesize
\item \verb|MetabolicModel|
\item \verb|ModelWrapper|
\end{itemize}
\end{column}
\begin{column}{.5\linewidth}
Accessors that work on all model types:
\begin{itemize}\footnotesize
\item \verb|reactions|, \verb|metabolites|
\item \verb|stoichiometry|
\item \verb|bounds|
\item \verb|coupling|, \verb|coupling_bounds|
\item \dots
\end{itemize}
\end{column}
\end{columns}
\end{frame}

\begin{frame}[standout]
Let's do a small demo of the method-based overloading mechanism\footnote{Not specific to Julia, you might know the same from C++, C\# TypeScript, Swift, Java, \dots} that makes this possible.
\end{frame}

\begin{frame}[fragile]{Systematization of Modifications and Analyses}
\begin{itemize}\footnotesize
\item \textbf{Model variants} are functions that takes a \verb|MetabolicModel| and produces another \verb|MetabolicModel| with some specific change applied.

Possible examples: \verb|with_added_reaction|, \verb|with_removed_metabolite|, \verb|knockout_gene|, \verb|with_changed_bounds|, \verb|with_enzyme_crowding_bounds|, \verb|with_enzyme_usage_bounds| \dots

\item \textbf{Analysis functions} take a \verb|MetabolicModel|, convert it into the optimizer language in some analysis-specific way, and run the analysis.

Examples: \verb|flux_balance_analysis|, \verb|parsimonious_flux_balance_analysis|, \verb|minimize_metabolic_adjustment|, \dots

\item \textbf{Optimizer modifications} are an additional concept added for efficiency, functions that take a model and its representation in the linear optimizer, and adjust the optimizer to do something differently.

Possible examples: \verb|knockout|, \verb|add_enzyme_usage_constraints|, \verb|change_objective|, \verb|set_quadratic_objective|, \verb|set_optimizer_parameter|, \verb|add_loopless_constraints|, \dots

\item \textbf{Meta-analysis functions} run analysis functions in multiple steps for a more complicated goal.

Main example: \verb|screen|, derived examples: \verb|objective_envelope|, \verb|flux_variability_analysis|, \dots
\end{itemize}
\end{frame}

\begin{frame}[fragile]{Systematization of Modifications and Analyses}
\framesubtitle{Simplified view}
\begin{itemize}
\item \textbf{Model variants} \uncover<2>{\alert{--- can be chained without any danger}}
\[\texttt{MetabolicModel} \to \texttt{MetabolicModel}\]
\item \textbf{Analysis functions} \uncover<2>{\alert{--- applied at the end of chain}}
\[\texttt{MetabolicModel} \to \text{result}\]
\item \textbf{Optimizer modifications} \uncover<4>{\alert{--- must be chained carefully}}
\[(\texttt{MetabolicModel}, \texttt{Optimizer}) \to \text{adjustments to the optimizer}\]
\item \textbf{Meta-analysis functions} \uncover<3>{\alert{--- apply to `fork' the chain}}
\[(\texttt{MetabolicModel}, \text{array of variants}, \text{analysis}) \to \text{array of results}\]
\end{itemize}
\end{frame}

\begin{frame}{Systematization example in a picture}
\framesubtitle{A complicated scheme of knockout screening}
\centering
\begin{tikzpicture}[lwd]
\begin{scope}[every node/.style={rotate=-90}]
\node[blk] (a) {load MAT}; \mkMdl{a}
\node[blk,above=0 of a] (b) {change objective}; \mkWrp{b}
\node[blk,above=0 of b] (c) {add pathway}; \mkWrp{c}
\node[blk,above=0 of c] (d) {\textcolor{permedcoe pink}{screen} $\forall {\color{green!66!gray}g}\in\textsc{Genes}$}; \mkWrp{d}
\node[blk,above=0 of d] (e) {knock out gene \color{green!66!gray}$g$}; \mkWrp{e}
\node[blk,above=0 of e] (f) {add loopless constraints}; \mkWrp{f}
\node[blk,above=0 of f] (g) {run pFBA}; \mkAna{g}
\end{scope}

\draw[permedcoe pink,decorate, decoration={brace, amplitude=2ex}] (e.south west) to node[midway, anchor=south, yshift=2ex, color=permedcoe pink] {repeated for each gene} (g.north west);
\end{tikzpicture}
\end{frame}

\begin{frame}{Systematization example in a picture}
\framesubtitle{Backfitting data to lab measurements}
\centering
\begin{tikzpicture}[lwd]
\begin{scope}[every node/.style={rotate=-90}]
\node[blk] (a) {load HDF5}; \mkMdl{a}
\node[blk,above=0 of a] (c) {change bounds}; \mkWrp{c}
\node[blk,above=0 of c] (d) {\textcolor{permedcoe pink}{screen}\\\small $\forall {\color{green!66!gray}g_{1,2}}\in\textsc{Genes}$\\$\forall {\color{cyan!66!gray}p} \in \textsc{MSMS}$}; \mkWrp{d}
\node[blk,above=0 of d] (e) {knock out genes \textcolor{green!66!gray}{$g_1$} and \textcolor{green!66!gray}{$g_2$}}; \mkWrp{e}
\node[blk,above=0 of e] (f) {add GECKO constraints}; \mkWrp{f}
\node[blk,above=0 of f] (f1) {change protein amount bounds\\to match \textcolor{cyan!66!gray}{$p$}}; \mkWrp{f1}
\node[blk,above=0 of f1] (g) {run FBA, return growth}; \mkAna{g}
\end{scope}

\draw[permedcoe pink,decorate, decoration={brace, amplitude=2ex}] (g.north east) to node[midway, anchor=north, yshift=-2ex, color=permedcoe pink] {repeated many times} (e.south east);

\node[below=1em of c.east, font=\footnotesize, align=center, text width=5em] (substr) {Substrate knowledge};
\draw[arr] (substr) to (c);
\node[below=4em of d.east, font=\footnotesize, align=center, text width=7em] (msms) {Mass spectrometry proteomics data};
\draw[arr] (msms) to (d);
\end{tikzpicture}
\end{frame}

\begin{frame}{The main goals today}
\begin{enumerate}
\item Get familiar with this system
\item Use it for realistic tasks
\item Write your own model variants/modifications
\end{enumerate}
\end{frame}

\section{Exercises --- Basics}

\begin{frame}[fragile]{Start up!}
Load the E.~coli toy model (or the big one, it doesn't really matter), and find optimal production.

You need:
\begin{itemize}
\item \verb|load_model|
\item \verb|flux_balance_analysis_dict|
\end{itemize}
\end{frame}

\begin{frame}[fragile]{Start up!}
\framesubtitle{Manual vs.~named model modification}

\begin{enumerate}
\item \verb|convert| the model to a \verb|StandardModel|, restrict oxygen intake and see what happens.
\item Restrict oxygen intake using \verb|change_constraint|
\item Can the model live completely without oxygen? (glucose?)
\end{enumerate}
\end{frame}

\begin{frame}[fragile]{Start up!}
\framesubtitle{Modifications combine easily}
How much CO\textsubscript{2} can the model produce given just $0.5 \frac{\text{mmol}}{\text{g}_{\text{DW}} \text{h}}$ oxygen?

(use \verb|change_objective|)

\bigskip
What if we restrict some other intakes too?
\end{frame}

\begin{frame}[fragile]{Simple meta-analyses}
\begin{itemize}
\item Use \verb|flux_variability_analysis_dict| to explore variability in the reaction fluxes.
\item \dots ensuring the model produces at least at 90\% of the original production.
\item \dots given the oxygen intake is restricted.
\item \dots and there is no glucose to ingest.
\item \dots and gene with ID \texttt{b0727} (\emph{sucB}) is knocked out. (use \verb|knockout|)
\end{itemize}
\end{frame}

\begin{frame}[fragile]{SW engineering corner}
\textbf{What \emph{are} these modification functions?}
\pause
Functions that return another functions.
\bigskip\footnotesize
\begin{itemize}
\item Functions are the best (and likely only) means to freely \emph{parametrize the functionality} of code.
\item Think about them as small recipes that the target function reads and knows what to do.
\item You can easily make your own recipes to implement more functionality!
\end{itemize}
\end{frame}

\begin{frame}[fragile]{Using functions as parameters}
\textbf{Exercise:} \verb|bounds| parameter of FVA is actually also a `recipe' --- FVA gives it the optimal objective value it found, and expects to receive the bounds it should place on the objective. Try using \verb|objective_bounds| or \verb|gamma_bounds| manually.

\bigskip
\textbf{Task:} run a FVA that explores model production at \emph{precisely} 66\% of the optimum capacity.
\end{frame}

\begin{frame}[fragile]{Custom optimizer modification}
Make a modification that uses \verb|change_constraint| to reduce limits of all reactions that require a certain gene by a given factor.

Example use:
\begin{verbatim}
flux_balance_analysis_dict(m, GLPK.Optimizer, modifications = [
  choke_gene_reactions("b0727", 0.5),
])
\end{verbatim}
\end{frame}

\section{Exercises --- Screening}

\begin{frame}[fragile]{Screening}
\verb|screen| function ``just'' runs an analysis over a huge amount of models generated from a single original model.

\bigskip
We found that pretty useful:
\begin{itemize}
\item parallelizes well
\item you don't need to write possibly complicated cycles manually
\item prevents programming errors
\end{itemize}
\end{frame}

\begin{frame}[fragile]{Screening}
\framesubtitle{Parameters of {\tt screen()}}
\begin{itemize}
\setlength{\itemindent}{0em}
\setlength{\leftskip}{5em}
\item model
\item[\tt analysis=] a function to run on the model (possibly many times)
\item[\tt args=] array of arguments to be passed to all function invocations
\item[\tt variants=] array of variant chains to be applied to the base model before it is given to the analysis function
\end{itemize}
\end{frame}

\begin{frame}[fragile]{Screening}
Use \verb|screen| and \verb|change_constraint|\footnote{you may use also \texttt{change\_bound}, but that only works with mutable models} to see how much biomass can be produced depending on variable oxygen availability.\footnote{normally we would use \texttt{objective\_envelope}, which is optimized for this purpose}
\end{frame}

\begin{frame}[fragile]{Screening}
\framesubtitle{Knocking out reactions}
Use \verb|reactions| and \verb|change_constraint| to disable reactions, see what happens.

\begin{itemize}
\item what happens after choking the ATPM reaction?
\item \dots and the biomass reaction?
\end{itemize}
\end{frame}

\begin{frame}[standout]
If everything went all right\footnote{the plan was naive and this timing is improbable}, we should have a break sometime around now.
\end{frame}

\begin{frame}[fragile]{Screening}
\framesubtitle{Knocking out genes}
Use \verb|knockout| to see what the model would look like without important genes.

\bigskip
\begin{itemize}
\item make a list of genes that make the organism unable to grow
\item \dots and that make the model completely infeasible
\end{itemize}
\end{frame}

\begin{frame}[fragile]{Screening}
\framesubtitle{Knocking out double genes}
Run a double gene knockout screening on the model:
\begin{itemize}
\item make a matrix of gene combinations (let's not worry about duplicities now)
\item careful about \verb|knockout| --- it is not associative!
\end{itemize}
\end{frame}

\begin{frame}[fragile]{Screening}
\framesubtitle{Choking all possible genes}
Use the previously created modification to see what happens if you choke each of the genes by 10\%, 20\%, 30\%, \dots
\end{frame}

\begin{frame}[fragile]{A quick note about parallelization}
COBREXA.jl was originally intended as a HPC runner of large screening-type analyses. Speeding up your analysis is thus quite trivial:

\begin{enumerate}\small
\item Add a few Julia ``worker'' processes to form a cluster
  \begin{itemize}\scriptsize
  \item locally, \verb|using Distributed; addprocs(10);|
  \item on HPC, you can e.g.: \verb|using Distributed,ClusterManagers; addprocs_slurm(100);|
  \end{itemize}
  \dots you may check the available workers IDs by \verb|workers|
\item load COBREXA and the solver everywhere: \verb|@everywhere using COBREXA, GLPK|
  \item pass in argument \verb|workers| to parallelizable meta-analysis functions, such as:
\begin{verbatim}flux_variability_analysis(m, GLPK.Optimizer, workers=workers())
screen(m, analysis=..., args=..., workers=workers())\end{verbatim}
  Since the problems are moreless embarassingly parallel, the analysis times should get (roughly) divided by worker count.
\end{enumerate}
\end{frame}

\section{Modifying the modifications}

\begin{frame}{How to make a model variant?}
\begin{enumerate}
\item Make a new structure to hold your model
\item Load any information you need from parameters and the wrapped model
\item Overload the accessors to provide the modified information
\end{enumerate}

Efficiency concerns:
\begin{itemize}
\item Source the least amount of stuff
\item Modify/reconstruct only the data you really need to
\item You can cache some precomputed items locally (models are immutable!)
\end{itemize}
\end{frame}

\begin{frame}[fragile]{\upshape\texttt{ModelWrapper}}
\verb|ModelWrapper| is a subtype of \verb|MetabolicModel| that you can use to
easily derive models from others.

\begin{verbatim}
struct MyTrivialModel <: ModelWrapper
  inner :: MetabolicModel
end

COBREXA.unwrap_model(x::MyTrivialModel) = x.inner

m = MyTrivialModel(load_model(...))
flux_balance_analysis(m, GLPK.Optimizer)
\end{verbatim}
\end{frame}

\begin{frame}[fragile]{Overriding parts of models}
\begin{verbatim}
struct ScaledModel <: ModelWrapper
  inner::MetabolicModel
  ratio::Float64
end

COBREXA.unwrap_model(x::ScaledModel) = x.inner
COBREXA.bounds(x::ScaledModel) =
  let (lbs, ubs) = bounds(x.inner)
    (x.ratio .* lbs, x.ratio .* ubs)
  end
COBREXA.balance(x::ScaledModel) = x.ratio * balance(x.inner)

m = ScaledModel(load_model(...), 0.5)
flux_balance_analysis(m, GLPK.Optimizer)
\end{verbatim}
\end{frame}

\begin{frame}[fragile]{Helpful bonus: Piping}
\begin{verbatim}
m = ScaledModel(load_model(...), 0.5)
\end{verbatim}
\dots is the same as\dots
\begin{verbatim}
m = load_model(...) |> ScaledModel(0.5)
\end{verbatim}
\end{frame}

\section{Exercises --- Custom variants}

\begin{frame}[standout]{Reminder}
Why do we want to make model wrappers rather than optimizer modifications?
\bigskip
\begin{itemize}
\item Models can be converted to SBML/JSON and saved.
\item Well-constructed model wrappers chain indefinitely without any risks.
\end{itemize} 
\end{frame}

\begin{frame}{Simple custom models}
\framesubtitle{OmniColi magically receives ATP}
Make a \emph{model wrapper} where ATP is constantly ``injected'' into otherwise normal model.

\bigskip
Possible approaches:
\begin{itemize}
\item change the balance vector
\item add a reaction
\end{itemize}
(What are the benefits and downsides of both approaches?)
\end{frame}

\begin{frame}{Simple custom models}
\framesubtitle{Reaction coupling utilization}
Make a model where glucose and oxygen uptake is constrained to be the same (by molar amount) as carbon dioxide output.

\bigskip
Possible approaches:
\begin{itemize}
\item add an extra reaction
\item use coupling
\end{itemize}
\end{frame}

\begin{frame}{How to structure the community models?}
\framesubtitle{Let's ferment some example bread}
\begin{tikzpicture}[lwd]
\begin{scope}[every node/.style={rotate=-45}]
\node[sblk] (m1) {load \emph{S.~cerevisiae}}; \mkMdl{m1}
\node[sblk] (m2) at (5em, 0) {load \emph{L.~lactis}}; \mkMdl{m2}
\node[sblk] (m3) at (10em, 0) {load~rye~flour enzymes}; \mkMdl{m3}
\end{scope}

\begin{scope}[every node/.style={rotate=-90}]
\node[blk] (a) at (16em,0) {make community}; \mkWrp{a}
\node[blk,above=0 of a] (b) {\textcolor{permedcoe pink}{screen}\\\small $\forall {\color{orange!80!gray}c}\in S^2$}; \mkWrp{b}
\node[blk,above=0 of b] (c) {constraint abundances to \textcolor{orange!80!gray}{$c$}}; \mkWrp{c}
\node[blk,above=0 of c] (d) {compute envelope of \\ \mbox{CO\textsubscript{2}\,\texttimes\,lactate}}; \mkAna{d}
\end{scope}

\draw[arr] (m3.north) to[out=45,in=180] (a.south);
\draw[arr] ($(m2.north)!.8!(m2.north west)$) to[out=45,in=170] ($(a.south)!.5!(a.south west)$);
\draw[arr] ($(m1.north)!.8!(m1.north west)$) to[out=30,in=160] (a.south west);
\end{tikzpicture}
\end{frame}

\begin{frame}{Making communities}
\framesubtitle{Mutant E.~coli communities!}

Make a model that contains several copies of the wrapped model, each with a different variant applied (e.g., with a knockout).

\begin{itemize}
\item simplifications:
  \begin{itemize}
  \item start with just N same models
  \item we can ignore genes, annotations and other properties for now
  \end{itemize}
\item community members can be otherwise completely independent
\end{itemize}
\end{frame}

\begin{frame}{Making realistic communities}
\framesubtitle{SteadyCom-style communities}

Combine the reaction-rate balancing wrapper with the community wrapper to make a biomass-stable community.

\begin{itemize}
\item All submodels should produce the same amount of biomass.
\item Bonus exercise: All submodels have their O\textsubscript{2} vs. CO\textsubscript{2} exchanges balanced (individually).
\end{itemize}
\end{frame}

\begin{frame}{Combining screenings with custom wrappers}
\framesubtitle{SteadyEnvelope}
Which glucose vs.~ammonia intake ratio is the best for our E.~coli?
\end{frame}

\frame[standout]{That's all for today.\par \fontsize{30pt}{30pt}\selectfont Time for questions?}

\begin{frame}{Where to go next?}
\begin{itemize}
\item There is more functionality ready or being implemented. \\Check out COBREXA.jl docs and examples!
\item If you make a useful wrapper, you can easily share it with the community.
\item Try a HPC (perhaps one from EuroHPC's) for a really huge analysis.
\end{itemize}
\end{frame}

\makelastslide{}

\end{document}
